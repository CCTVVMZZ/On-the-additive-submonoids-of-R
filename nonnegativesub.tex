% -*- mode: latex; eval: (flyspell-mode 1); ispell-local-dictionary: "american"; TeX-master: "SubsemiR"; -*-

\begin{theorem}[\cite{AbelsManoussos2012}] \label{thm:additive-closure}
  Let $M$, $X \subseteq \Rnneg$ be such that $M$ is the additive monoid generated by~$X$.
  The intersection of $X$ and $[0, 1]$ is finite
  if, and only if,
  the intersection of $M$ and $[0, 1]$ is finite.
\end{theorem}

\begin{proof}
  The ``if part'' holds true because $X \cap [0, 1] \subseteq M \cap [0, 1]$. 
  It remains to prove the ``only if part''.
  Since replacing $X$ with $X \setminus \{ 0 \}$ leaves $M$ unchanged,
  we may assume $0 \notin X$ without loss of generality.
  % Since $b = 0$ implies $M \cap [0, 1] = \{ 0 \}$,
  % we may also assume $b > 0$.
  Assume that $X \cap [0, 1]$ is finite.
  Then, there exists $m \in \bNast$ such that $m^{-1}$ is a lower bound for~$X$.
  To prove that $M \cap [0, 1]$ is finite,
  we check that each element of $M \cap [0, 1]$ can be written as the sum of at most $m$ elements of $X \cap [0, 1]$.
  Let $v \in M \cap [0, 1]$.
  Since $v \in M$, $v$ can be written as a finite sum of elements of $X$,
  and thus there exist
  a finite subset $U \subseteq X$ and
  a function $\lambda \colon U \to \bNast$ such that
  $v = \sum_{u \in U} \lambda(u) u$.
  For each $u \in U$, we have $u \le \lambda(u) u \le v \le 1$,
  whence $U \subseteq X \cap [0, 1]$.
  Therefore, $v$ is the sum of $\sum_{u \in U} \lambda(u)$ elements of $X \cap [0, 1]$.
  In addition, we have $1 \le m u$  for each $u \in X$, whence  
  $$
  \sum_{u \in U} \lambda(u) =
  \sum_{u \in U} \lambda(u) \cdot 1
  \le  \sum_{u \in U} \lambda(u) m u =  m v \le m \, . 
  $$
\end{proof}

Let us briefly comment the statement of Theorem~\ref{thm:additive-closure}.
First, $[0, 1]$ can be harmlessly replaced by any bounded interval that contains $0$ as an element.
Second, replacing $[0, 1]$ with, say, $[1, 2]$ makes the theorem go wrong:
in the case where $X = [0, 1]$ and $M = \Rnneg$,
$X \cap [1, 2] = \{ 1 \}$ is finite whereas $M \cap [1, 2] = [1, 2]$ is not.

\begin{theorem} \label{thm:discrete-dense}
  There exists an additive submonoid $M$ of $\bQ \cap \Rnneg$ such that
  every bounded subset of $M$ is finite and
  $M - M$ is dense in~$\bR$.
\end{theorem}

\begin{proof} 
  Put
  $$
  X_n = \bZ 2^{-n} \cap \left[n, \infty \right[
  $$ for each $n \in \bN$ and
  set $M = \bigcup_{n \in \bN} X_n$.
  First, inclusion $M \subseteq \bQ \cap \Rnneg$ is clear.
  Second,
  we have
  $0 \in X_0$ and
  $$X_m + X_n \subseteq  X_{\max \{ m, n \}}$$ for all $m$, $n \in \bN$,
  so $M$ is an additive monoid. 
  Third,
  we have
  $$
  2^{- n} = (n + 2^{- n}) - n  \in X_n - X_0  \subseteq  M - M
  $$
  for each $n \in \bN$.
  % for each $n \in \bN$,
  % we have
  % $n + 2^{- n} \in X_n$ and $n \in X_0$,
  % whence $2^{- n} = (n + 2^{- n}) - n  \in M - M$.
  Therefore, it follows from Lemma~\ref{lem:rateau} that $M - M$ is dense in~$\bR$.
  Fourth and last, let $k \in \bN$.
  We have
  $$X_n \cap [0, k] =  \seg{n 2^n}{k 2^n} 2^{- n}$$
  for every $n \in \bN$,  whence
  $$
  M \cap [0, k] = \bigcup_{n = 0}^k X_n \cap [0, k] \,, 
  $$
  and consequently, $M \cap [0, k]$ is finite.
\end{proof}